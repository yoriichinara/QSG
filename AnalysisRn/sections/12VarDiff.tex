\section{Variedades Differenciáveis}

\begin{definition}
    Sejam \(X\subseteq \R^n\) e \(Y\subseteq \R^m\), uma aplicação \(F\in C^k(X;Y)\) sse \(\forall p \in X\), \(\exists \widetilde{F} \in C^k(\U_p; \R^m)\) tal que \( \widetilde{F}|_{_{\U_p \cap X}} = F\).    
\end{definition}

%\begin{note}
%    Análogamente, se \(F \in C^\infty(X;Y)\), então dizemos que é \emph{suave}.
%\end{note}

\begin{definition}
    \(F: X \to Y\) é \emph{diffeomorfismo} de classe \(C^k\) se \(F\in C^k(X;Y)\) é homeomorfimso e \(F^{-1}\in C^k(Y;X)\). Denotamos \(^k F : X \simeq Y \). 
\end{definition}
\begin{example}
    Sejam \(X= \{(x,y) \in \mathbb{S}^1: y > 0 \}\) e \(Y = (-1,1)\). A aplicação \(\pi: X \ni (x,y) \mapsto x \in Y\) é \(^\infty F: X\simeq Y \). \textcolor{gray}{\(\rightarrow\) Verifique as condições}
\end{example}
\begin{definition}
    Um subconjunto \(^k_d M \subseteq \R^n\) é uma \emph{subvariedade} de classe \(C^k\) e dimensão \(d\), se  \(\forall p \in M \), \(\exists\ ^k \varphi: \U_p \cap M \simeq V\ab \subseteq \R^d\) \emph{carta local}, cujas componentes \(\varphi = (\x_1, \ldots, \x_d)\) são \emph{coordenadas locais} de \(M\) em \(p\). A inversa \(\varphi^{-1} = \psi : V \to \U_p \cap M \) é uma \emph{parametrização local} da variedade. 
\end{definition}

\begin{note}
    Por simplicidade na notação usaremos \(\U_p\) para indicar vizinhanza de \(p\) em \(M\) como subespaço de \(\R^n\). 
\end{note}

\begin{example}
    \(^{\infty}_{\hspace{0.15cm}n} U\ab\subseteq \R^n\). \textcolor{gray}{\(\rightarrow \) Tome \(\varphi = \id_n\)}
\end{example}
\begin{example}
    \(^{\hspace{0.3cm}\infty}_{n-1}\E^{n-1} \subseteq \R^n\). \textcolor{gray}{\(\rightarrow\) Construa a projeção estereográfica} 
\end{example}

\begin{lemma}
    Sejam \(^k_d M \subseteq \R^n \), \(V\ab\subseteq \R^d\) e \(\psi: V \to \R^n\) tal que \(\psi(V)\subseteq M \). Então, \(\psi\) é uma paremetrização local de \(M\) sse \(\psi \) é imersão e homeomorfismo (\(\tau_{_M}\)). 
\end{lemma}

\demo{(\(\Leftarrow\)) A bijeção é imediata de \(\psi(V)\ab \subseteq M\). Aplique \(\blacksquare\) (FLI) e faça a composição com a projeção. }

\alerta{\centering Imersão injetiva \textcolor{red}{\(\not\Rightarrow \)} homeomorfismo}

\begin{example}%[Coordenadas Esféricas]
    Parametrização local da esfera (coordenadas esféricas). Seja \(\psi: [0,\pi]^{n-2}\times [0, 2\pi) \to  \R^n\) tal que     
    \[(\theta_i) \mapsto \left(\cos\theta_1, \ldots, \prod^{j<i}\sin\theta_j \cdot \cos\theta_{i},\ldots, \prod\sin\theta_i\right). \] 
\end{example}
\begin{example}%[Gráficos]
    Se \(F\in C^k(U; \R^m)\) então \(\operatorname{graf}(F):= \{(\textbf{x},F(\textbf{x})) \in U\times\R^m\} = \ ^k_n M \subseteq \R^{n+m}\), com \(\psi: \textbf{x} \mapsto (\textbf{x}, F(\textbf{x}))\) e \(\varphi = \pi : (\textbf{x}, F(\textbf{x})) \mapsto \textbf{x}\).   
\end{example}
\begin{example}
    Seja \(^k H\subseteq \R^n\) hipersuperficie, então \(^{\hspace{0.5cm}k}_{n-1} H\subseteq \R^n\). \textcolor{gray}{\(\rightarrow\) \(H\) é localmente o gráfico de uma função} 
\end{example}

\begin{proposition}
    Seja \(G\in C^k(U;\R^m) \). Se \(\forall p \in U, \ G\) é submersão em \(p\), então \[\{p \in U : G(p)=0\} =:\ ^{\hspace{0.65cm}k}_{n-m}M \subseteq \R^n.\]     
\end{proposition}

%\begin{proposition}
%    \textcolor{rojoscuro}{\underline{\textbf{Exercise}}} \ Muestre que \(^k_d M\subseteq \R^n\) sii \(\forall p \in M, \ \exists \ ^kF: \U_p\simeq V\ab \) tal que \(F(\U_p\cap M ) = V \cap (\R^d \times \{0\}) = \{\textbf{y}\in V: \textbf{y} = (\textbf{y}^{(r)}, 0)\}\). 
%\end{proposition}

\begin{definition}
    Sejam \(^k_dM\subseteq \R^n\) e \(^k \psi_a: V\ab \simeq \U_p\) uma parametrização local tal que \(\psi_a(a)=p\). O \emph{espaço tangente} a \(p \in M \) é o dado por 
    \[T_pM := \operatorname{Im}(D\psi_a(a))= \bigoplus \R \frac{\partial \psi_a}{\partial x_i} (a)\leq \R^n. \]
\end{definition}

\begin{note}
    O espaço afim \(p + T_pM\) é o "intuitivamente" tangente a \(M\) en \(p\).
\end{note}

\begin{proposition}
    Sejam \(^k_d M \subseteq \R^n\) e \(c \in C^k(\V_\epsilon ; M)\) caminho em \(M\), então 
    \[T_pM = \{c'(0): c(0) =p \}. \] 
\end{proposition}

\begin{note}
    O espaço tangente não depende da parametrização.
\end{note}

\begin{example}
    Se \(U\ab\subseteq \R^n\) então \(T_pU= \R^n\). 
\end{example}
%\begin{example}
%    Se \(\psi(\alpha_1, \alpha_2)= p \in \E^2\) entonces  
%    \(T_p\E^2 = \R \frac{\partial \psi}{\partial \theta_1}(\alpha_1, \alpha_2)\oplus \R \frac{\partial \psi}{\partial \theta_2}(\alpha_1, \alpha_2)\). 
%\end{example}

%\begin{exercise}
%    Determine una parametrización del toro \(\mathbb{T}\subseteq\R^3\) y cálcule \(T_p\mathbb{T}\). 
%\end{exercise}

\begin{note}
    Em diante, para refererinos a \(^{\hspace{0.1cm}k}_{d_1}M\subseteq \R^n\) e \(^{\hspace{0.1cm}k}_{d_2}N\subseteq \R^m\) escrevemos simplesmente \(M\) e \(N\). %Dimensão, classe e espaço ambiente serão descritos apenas se fora necessario.
\end{note}

\begin{proposition}
    \(F\in C^k(M;N)\) sse \(\forall \psi: V\ab \to \U_p\) parametrização local de \(M\) em \(p\), tem-se que \(F \circ \psi \in C^k(V; N)\). \textcolor{gray}{\(\rightarrow\) Basta ver só um  atlas}  
\end{proposition}

\begin{definition}
    Seja \(F\in C^k(M;N)\). A derivada de \(F\) em \(p \in M\) é a aplicação linear \(DF(p): T_pM \to T_{F(p)}N \) tal que \( v\mapsto D\widetilde{F}(p) \cdot v\). 
\end{definition}

\begin{lemma}
    Se \(F\in C^k(M;N)\) então derivada \(DF(p)\) não depende da \(\widetilde{F}\).  
\end{lemma}

\demo{\parbox[a]{0.28\linewidth}{ }\parbox[b]{0.72\linewidth}{\vspace{0.3cm}Tome \(\psi_a\) e \(\psi_{b}{'}\) parametrizações locais de \(p\in M\) e \(F(p) \in N\). Faça \(\varphi{'} \circ F \circ \psi: V \to V{'}\), pelo \(\blacksquare\) (Regla da Cadeia) temos o diagrama a dereita. Note que \(D\widetilde{F}: \operatorname{Im}(D\psi_a(a))\mapsto \operatorname{Im}(D\psi_{a{'}}(a{'}))\). \vspace{0.3cm}}
}

\begin{tikzpicture}[scale = 1.2, >=to, shift={(11.95cm,1.5cm)}, baseline={(current bounding box.center)}, remember picture, overlay]
    
    \node (Up) at (0,0) { $\R^{d_1}$};
    \node (Rn) at (2.5,0) { $\R^{d_2}$};
    \node (Rn2) at (2.5,2) { $\R^n$};
    \node (U) at (0,2) { $\R^n$};
    
    %\node (c) at (3.6,0) {\footnotesize $L$};
    %\node (d) at (3.6,2) {\footnotesize$L^{-1}$};
    %\node (c1) at (3.6,1) {\rotatebox{90}{$\longmapsto$}};

    \draw[->] (Up) -- node[above, xshift = -0.45cm, yshift = -0.2cm] {\scriptsize \(\psi\)} (U);
    \draw[->] (Up) -- node[below] {\scriptsize \(D(\varphi{'}\circ F \circ \psi)\)} (Rn);
    \draw[->] (U) -- node[above] {\scriptsize \(DF\) } (Rn2);
    \draw[->] (Rn) -- node[right, xshift = 0.05cm, yshift = 0cm] {\scriptsize \(\psi{'}\)} (Rn2);
    \draw[->, color = gray, >=to] (1.15,0.85) arc (240:-60:0.25);

\end{tikzpicture}

\vspace{-1cm}

\begin{example}
    \(F: \E^1\to \E^1\) tal que \( (\cos\theta, \sin\theta) \equiv (x,y) \mapsto (x^2-y^2, 2xy) \equiv (\cos 2\theta, \sin 2\theta)\). \textcolor{gray}{\(\rightarrow \) Faça as contas }
\end{example}
%\begin{exercise}
%    Sea \(c \in C^k(\V_\epsilon; \R^m)\) tal que \(c(0)=p\). Muestre que si \(\vec{v} = c'(0)\in T_pM \) entonces \(DF(p)\cdot c{'}(0)= (F\circ c){'}(0)\). 
%\end{exercise}

\begin{note}
    Em geral, tudo o que foi visto para aplicações, vale em variedades: 
    \begin{description}
        \item[Regla da Cadeia] Se \(F\in C^k(M;N)\) e \(G \in C^k(N;P)\) então \(G \circ F \in C^k(M;P)\) e \(D(G\circ F)(p)  = DG(F(p))\cdot DF(p)\). 
        \item[Derivada da Identidade] \(D\id_{M} = \id_{T_pM}: T_pM \to T_pM\).  
        \item[Teorema da Função Inversa] Seja \(F\in C^k(M;N)\) tal que \(D\widetilde{F}(p)\) é isomorfismo, então \(\exists \ ^k F : \U_p\cap M \simeq \mathcal{V}_{F(p)} \cap N\).  
    \end{description}
\end{note}

\subsection{Valores Regulares}

\begin{definition}
    Seja \(F \in C^k(M;N)\). Um ponto \(p\in M\) é \emph{regular} se \(F\) é submersão em \(p\), se não fora regular então é \emph{ponto crítico} de \(F\).  
\end{definition}

\begin{note}
    Se \(\exists p\in M\) ponto regular de \(F\) então \(\dim M \geq \dim N\).
\end{note}

\ejemplo{ \centering \(\blacksquare\) (Multiplicadores de Lagrange)
\tcblower
\begin{example}
    Seja \(M= \ ^k H\subseteq \R^n\) hipersuperficie definida por \(g(\x)=0\) e \(f: \R^n \to \R\) função escalar, então 
    \[p \in M \text{ ponto crítico de } f(M) \textcolor{verdeoscuro}{\ \Leftrightarrow\  }df(p): T_pM \not \twoheadrightarrow \R.\]
\end{example}
}

%\{El multiplicador de Lagrange, de nuevo }

\begin{definition}
    Seja \(F\in C^k(M;N)\). Um valor \(q\in N\) é \emph{valor crítico} se \(\exists p\in M\) tal que \(p \) é ponto crítico de \(F\) e \(F(p)= q\), de outra forma é um \emph{valor regular} de \(F\).  
\end{definition}

\alerta{ \centering \(p\in M\) regular \textcolor{red}{\(\nRightarrow\)} \(q = F(p)\) regular}

\begin{note}
    Um conjunto da forma \(F^{-1}(q)\) é chamado \emph{fibra}.% ou "conjunto de nível". 
\end{note}

\begin{theorem}[\emph{Fibras Regulares}]
    Sean \(^{\hspace{0.1cm}k}_{d_1} M\subseteq \R^n\), \(^{\hspace{0.1cm}k}_{d_2}N\), \(F\in C^k(M;N)\). Si \(q \in N\) es un valor regular de \(F\), entonces \(P = F^{-1}(q)\) es \(^{\hspace{0.65cm}k} _{d_1 - d_2}P \subseteq M  \subseteq \R^n\).
\end{theorem}

\demo{Sejam \(p = F^{-1}(q)\) e \(d_1 - d_2 = c \). Note que \(\R^c \cong \ker DF(p) \leq \R^n\), defina \(L: \R^n \to \R^c\) tal que \(L\cdot \ker DF(p)\) é isomorfirsmo. Faça \(G:M \to N \times \R^c\) tal que \(\x\mapsto (F(\x), L(\x))\). Aplique \(\blacksquare\) (TFI) e conclua. }

\begin{example}
    \(1\) é valor regular de \(g: \R^n \ni \textbf{x}\mapsto  \sum x_i^2 \in \R\) e \(g^{-1}(1)= \E^{n-1}\). 
\end{example}

\ejemplo{ \centering Variedades de dimensão \(0\) 
\tcblower 
\begin{example}
    Sejam \(F\in C^k(M;N)\) com \(\dim M = \dim N \) e \(q \in N \) valor regular de \(F\). Então, \(P = F^{-1}(q)\) é um conjunto discreto. Ainda mais, se \(M\) fora compacto \(\Rightarrow |P| < \infty\). \textcolor{gray}{\(\rightarrow \ \R^0 = \{0\} \) }
\end{example}
}


%\begin{exercise}
%    Sea \(\mathcal{O}(n) := \{A \in GL_{n}(\R) : A A^T = \id_{n}\}\). Muestre que \(^{\hspace{0.8cm}k}_{n(n-1)} \mathcal{O}(n) \subseteq \R^{n^2}\). Cálcule \(T_{\id_{n}}(\mathcal{O}(n))\) y verifique sin es un conjunto conexo. 
%\end{exercise}

%\begin{exercise}[Fibras de Rango Constante]
%    Sea \(F\in C^k(M;N)\) e \(F^{-1}(q) \subseteq \ ^{\hspace{0cm}k}_{d}M\subseteq \R^n\) tal que \(\exists \ \U_{F^{-1}(q)}\) donde \(\operatorname{ran}(DF(\textbf{x})) = r\), entonces \(^{\hspace{0.35cm}k}_{d - r}F^{-1}(q)\subseteq M \subseteq \R^n\). 
%\end{exercise}
%\begin{note}
%    ¿Que se puede decir de los valores regulares?
%\end{note}

\begin{theorem}[\emph{Brown-Sard}]
    Seja \(F\in C^k(M;N)\). Se \(k \gg 0\) então o subconjunto de valores regulares de \(F\) é denso em \(N\).% \footnote{Se exige \(k\geq \dim M - \dim N + 1\) pero así se ve mas bonito }
\end{theorem}

\begin{note}
    A medida de \(B= (a_i,b_i)^n\subseteq \R^n\) bloque aberto é \(\mu(B):= \prod (b_i-a_i)\).
\end{note}

\begin{definition}
    Um subconjunto \(S\subseteq \R^n\) tém \emph{medida nula} se \(\forall \epsilon >0\),  \(\exists B_i\) bloques abertos tais que  \(S\subseteq \bigcup B_i\) e \(\sum \mu(B_i)< \epsilon\). .  
\end{definition}

\begin{example}
    Se \(n<m\) então \(\R^n\times \{0\}\subseteq \R^m\) tém medida nula.   
\end{example}
\begin{example}
    \(S\subseteq \R^n\) enumerável tém medida nula.  \textcolor{gray}{\(\rightarrow\) Pontos isolados} 
\end{example}
\begin{theorem}[\emph{Sard}]
    Sejam \(U\ab\subseteq \R^n\) e \(F\in C^k(U; \R^m)\) tal que \(k\geq \{n-m+1, 1\}\). Então, o conjuntos de valores críticos de \(F\) tém medida nula em \(\R^n\). 
\end{theorem}

\begin{example}
    Pelo \(\blacksquare\) (Sard) \  \(^k_d M\subseteq \R^n\) com \(d<n\) tém medida nula. \textcolor{gray}{\(\rightarrow\) Todos os valores da inclusão \(\iota : M \hookrightarrow \R^n\) são críticos}
\end{example}

\begin{exercise}
    Se \(S\subseteq \R^n \) tém medida nula então \(\R^n\setminus S \) é denso. 
\end{exercise}

