\section{Desigualdade do Valor Medio} 
\begin{lemma}
    Sejam \(f:U\ab\subseteq \R^n \to \R \), \(p\in U\) e \(\vec{v} \in \R^n\) tais que \([p,p+v]\subseteq U\). Se \(\forall t \in (0,1)\) a função \(\varphi : t \mapsto f(p+tv)\) é diff em \(p+tv\), então \(\varphi\) é diff em \(t\) e \(\varphi{'} (t) = df(p+tv)\cdot v= \frac{\partial f }{\partial v} (p+tv)\). \textcolor{gray}{\(\rightarrow \) Faz \( h\to 0 \) de \(\varphi(t+h)\)}
\end{lemma}


\begin{theorem}[\emph{TVM - \(\R^n\)}]
    Seja \(f:U\ab\subseteq \R^n\to \R\) continua em \([p,p+v]\) e diff em \((p, p+tv)\), então \(\exists\theta \in (0,1)\) tal que \(f(p+v)-f(p)= df(p+\theta v )\cdot v \). 
\end{theorem} 
\demo{Tome \(\varphi(t) = f(p+tv)\), aplique o lema acima e TVM.}

\begin{corollary}[\emph{DVM - \(\R^n\)}]
    Sejam \(f: U\ab\subseteq \R^n \to \R\) diff, \(K\subset U\) convexo e \( c\geq 0 \) tal que \(\forall \textbf{x}\in K, \ \|df(\textbf{x})\|\leq c\). Então, \(\forall p,q\in K \) temos  \(\|f(p)-f(q)\|\leq c \|p-q\|\). \textcolor{gray}{\(\rightarrow\) Tome \(q = p+v\) no Teorema anterior} 
\end{corollary}

%\begin{exercise}
%    Muestre que si \(f\in C^1(U)\) y \(K\) es compacto, entonces la hipótesís del colorario es automaticamente satisfecha para algún \(c\geq 0\).  \textcolor{gray}{\(\rightarrow\) Bolzano-Weierstrass} 
%\end{exercise}
\newcommand{\Li}{\mathcal{L}}
\begin{definition}
    Seja \(\ell \in \Li(\R^n; \R)\). Definimos \(\displaystyle \|\ell \| := \sup_{\|v\|=1} |\ell v |\).  
\end{definition}

\begin{exercise}
    Se \(\exists \vec{w}\in \R^n\) tal que \(\ell v = \langle w, v \rangle \), então \(\|\ell\| = \|w\|\). Em particular, se \(f\) é diff em \(p\), então \(\|df(p)\|= \|\nabla f (p)\|\). \textcolor{gray}{\(\rightarrow \) Use Cauchy-Schwarz para provar as duas desigualdades} 
\end{exercise}

\begin{definition}
    Uma função \(f:X\subset \R^n\to \R \) é \emph{Lipschitz continua} se \(\exists c\geq 0\) tal que \(\forall p,q \in X, \ \|f(p)-f(q)\|\leq c \|p-q\|\). 
\end{definition}
\begin{corollary}
   \(\blacksquare\) TVM - \(\R^n\) implica \(f\) Lipschitz em \(K\). 
\end{corollary}

\Ei



