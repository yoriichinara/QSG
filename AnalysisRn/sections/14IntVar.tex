\section{Integração em Variedades}

\begin{note}
    Nesta seção \(I = [0,1]\subset \R\). 
\end{note}

\subsection{Integração em Cadeias}

\begin{definition}
    Um \(k\)\emph{-bloco singular} em \(M\) é uma aplicação \(\sigma \in C^\infty(I^k; M)\).
\end{definition}

%\ejemplo{ \centering  
%    \tcblower
\begin{example}[$n$-bloco padrão em \(\R^n\)]
    \(\id_{I^n} : I^n \to I^n \subset \R^n \). 
\end{example}
%}
\begin{example}%[$n$-bloco padrão em \(\R^n\)]
    Uma curva suave \(c : I \to M \) é um \(1\)-bloco singular. 
\end{example}

\alerta{
\centering \(\sigma \  \ k\)-bloco singular \(\textcolor{red}{\neq}  \ \ \operatorname{Im}(\sigma)\subseteq M\) 
\tcblower
\begin{example}
    \(\sigma: I^n \ni \x \mapsto p \in M \) é um $n$-bloco singular.%, chamamos ele em particular de $n$\emph{-cubo singular}. 
\end{example}
}

\begin{definition}
    Sejam \(\omega \in \Omega^k(M)\) e \(\sigma \) um $k$-bloco singular. Então,  
    \[\int_\sigma \omega = \int_{I^k} \sigma^*\omega = \int_{I^k}f(\x) \ dx_1\wedge \cdots \wedge dx_k =\int_{I^k} f(\x) \ d\x. \]  
\end{definition}

\ejemplo{
\centering Integral de linha
\tcblower
\begin{example}
    Sejam \(c(t) : [a,b] \to U\ab\subseteq \R^n\) curva suave e \(\overset{n}{\sum} f_i \ dx_i = \omega \in \Omega^k(U) \), então temos
    \[\oint_c \omega = \int_a^b c^*\omega \ dt = \int_a^b \sum^n (f_i \circ c)(t) c_i{'}(t) \ dt\]
\end{example}
}

%\parbox[l]{0.6\linewidth}{
\begin{definition}
    Uma $k$\emph{-cadeia} de blocos singulares em \(M\) é uma expresão da forma 
    \[ c = \overset{k}{\sum} n_i \sigma_i, \ n_i \in \mathbb{Z} \ \text{e } \sigma_i:I^k\to  M.\] 
    %onde \(n_j \in \mathbb{Z} \) e \(\sigma_j: I^k \to M \) é um $k$-cubo singular.
\end{definition}
%}
\vspace{0.3cm}
\begin{example}
 \(c = 2c_1 + c_2 \textcolor{red}{-} c_3\).    
\end{example}

\- \vspace{-1cm}
\begin{tikzpicture}[>=to, shift={(8.25cm,-1.1cm)}, baseline={(current bounding box.center)}, remember picture, overlay]
  % Dibuja la cuadrícula de puntos

  % C1
  \draw[thick] (0.5,2.5) to[out=-20, in=180] (2.5,3) node[yshift = 3cm, xshift=1cm, midway, above] {$c_1$};
    \fill (0.5,2.5) circle (2pt);
  % C2
  \draw[thick] (2.5,3) to[out=-45, in=170] (4,2) node[yshift = 2.5cm, xshift=3.7cm, midway, above] {$c_2$};
    \fill (2.5,3) circle (2pt);
    \fill (4,2) circle (2pt);

  % C3
  \draw[red] (5,2.5) to[out=60, in=225] (5.7,3.5) node[yshift = 2.3cm, xshift=6cm, midway, above] {\textcolor{black}{$c_3$}};
    \fill (5.7,3.5) circle (2pt);
    \fill (5,2.5) circle (2pt);

    \node at (1.6,2.7) {\rotatebox{33}{$>$}};
    \node at (1.4,2.6) {\rotatebox{33}{$>$}};
    \node at (3.12,2.4) {\rotatebox{-39}{$>$}};

    \node[red] at (5.35,3.05) {\rotatebox{230}{$>$}};

\end{tikzpicture}

\begin{note}
    Mais formalmente, as $k$-cadeias são elementos do grupo abeliano livre gerado pelos $k$-blocos singulares. 
\end{note}
\vspace{-0.3cm}
%\parbox[l]{0.3\linewidth}{\ }
\begin{definition}
    Se \(\omega \in \Omega^k(M)\) e \(c \) é uma $k$-cadeia singular, então \(\displaystyle \int_c \omega =  \sum^k n_i\int_{\sigma_i} \omega .\)  
\end{definition}

\vspace{-0.3cm}
\begin{definition}
    Seja \(\sigma \) um $k$-bloco singular. O \emph{borde} de \(\sigma\) é a \((k-1)\)-cadeia dada por
    \- \vspace{-0.2cm}
    \[\hspace{5cm} \partial \sigma = \sum^k (-1)^i (\sigma_{(i,0)}- \sigma_{(i,1)}).\]
\end{definition}

\begin{tikzpicture}[scale= 1, >=to, shift={(2.9cm,0.3cm)}, baseline={(current bounding box.center)}, remember picture, overlay]

  % Coordenadas de los vértices del cuadrado
  \coordinate (A) at (-0.5,1);
  \coordinate (B) at (0.5,1);
  \coordinate (C) at (0.5,2);
  \coordinate (D) at (-0.5,2);

  % Relleno del área interior
  \fill[gray!13] (A) -- (B) -- (C) -- (D) -- cycle;

  % Cuadro con flechas en los bordes
  \draw[thick] (A) -- (B); %node[midway, below] {\small $\sigma_{(2,0)}$};
  \draw[thick] (B) -- (C) ;%node[midway, right] {\small $\sigma_{(1,1)}$};
  \draw[thick] (D) -- (C) ;%node[midway, above] {\small $\sigma_{(2,1)}$}; % misma etiqueta que la derecha (según imagen)
  \draw[thick] (A) -- (D) ;%node[midway, left] {\small $\sigma_{(1,0)}$};
  \node[thick] at ($(A) + (2,0.5)$) {$ \longrightarrow $};
  \node[thick] at ($(A) + (2,1)$) {$ \partial $};
  \node[thick] at ($(A) + (0.5,0)$) {\small\rotatebox{0}{$>$}};
  \node[thick] at ($(D) + (0.5,0)$) {\small\rotatebox{0}{$>$}};
  \node[thick] at ($(B) + (0,0.5)$) {\small\rotatebox{90}{$>$}};
  \node[thick] at ($(A) + (0,0.5)$) {\small\rotatebox{90}{$>$}};

  % Puntos negros en los vértices
  \foreach \point in {A,B,C,D}
    \fill[black] (\point) circle (1.5pt);


% Coordenadas de los vértices del cuadrado
  \coordinate (A) at (2.5,1);
  \coordinate (B) at (3.5,1);
  \coordinate (C) at (3.5,2);
  \coordinate (D) at (2.5,2);

  % Relleno del área interior
%  \fill[gray!13] (A) -- (B) -- (C) -- (D) -- cycle;

  % Cuadro con flechas en los bordes
  \draw[thick] (A) -- (B) ;%node[midway, below] {\small $\sigma_{(2,0)}$};
  \draw[thick] (B) -- (C) ;%node[midway, right] {\small $\sigma_{(1,1)}$};
  \draw[thick] (D) -- (C) ;%node[midway, above] {\small $\sigma_{(2,1)}$}; % misma etiqueta que la derecha (según imagen)
  \draw[thick] (A) -- (D) ;%node[midway, left] {\small $\sigma_{(1,0)}$};

%  \node[thick] at ($(A) + (-0.75,2.5)$) {\small $M$};
% \draw[thick, ->] ($(A) + (-1.75,2)$) to[in=180, out=90] ($(A) + (-1.25,2.5)$);
%  \draw[thick, ->] ($(A) + (0.5,2)$) to[in=0, out=90] ($(A) + (-0.25,2.5)$);

  \node[thick] at ($(A) + (1.85,0.5)$) {$\sim $};
  \node[thick] at ($(A) + (0.5,0)$) {\small\rotatebox{0}{$>$}};
  \node[thick] at ($(D) + (0.5,0)$) {\small\rotatebox{180}{$>$}};
  \node[thick] at ($(B) + (0,0.5)$) {\small\rotatebox{90}{$>$}};
  \node[thick] at ($(A) + (0,0.5)$) {\small\rotatebox{270}{$>$}};

  % Puntos negros en los vértices
  \foreach \point in {A,B,C,D}
    \fill[black] (\point) circle (1.5pt);
\end{tikzpicture}

\vspace{-1cm}
\begin{exercise}
    Se \(c\) é uma $k$-cadeia, então \(\partial (\partial c) = 0 \). \textcolor{gray}{\(\rightarrow\) \(\partial^2 = 0\), olha no Spivak}  
\end{exercise}

\begin{theorem}[\emph{Stokes}]
    Seja \(\omega \in \Omega^{k-1}(M)\) e \(c\) $k$-cadeia em \(M\), então
    \[\int_c d\omega = \int_{\partial c } \omega.\]
\end{theorem}