\section{Multilinearidade}

Em diante entendemos \(\Pi V_j = V_1 \times V_2 \times \cdots \times V_k\) como o espaço vetorial dado pelo produto cartesiano dos \(\F\)-espaços vetorias \(V_j\) com \(j\leq k \in \N\). 

\begin{definition}
    Sejam \(\Pi V_j\) e \( W\) \(\F\)-espaços vetorias. Denotamos por \( \mathcal{F}\left(\Pi V_j, W\right)\) o conjunto de todas as funções \(\phi : \Pi V_j \to W \).    
\end{definition}

\begin{definition}
    Uma função \(\phi \in \mathcal{F}( \Pi^k V_j, W)\) é $k$\emph{-linear} se é linear em cada entrada. Ou seja, \(\forall j\leq k\), \(\forall (v_{i_0}) \in \Pi V_j \) fixo \(: v_{j_0} =0\), \(\forall(v_i), (v_i') \in \Pi V_j\) e \(\forall \lambda \in \F\), 
    \[  \textstyle \phi_j ((v_{i_0}) + (v_i+ \lambda v_i')_{_{\delta_j}}) = \phi ((v_{i_0})+ (v_i)_{_{\delta_j}}) + \lambda \phi((v_{i_0})+(v_i')_{_{\delta_j}}).\footnote{Entendase \((v_i)_{_{\delta_j}}:= (v_i\cdot \delta_j)\).}\]  
\end{definition}

\vspace{-0.5cm}
\begin{definition}
    \centering \(\hom^k(\Pi V_j,W):= \{\phi \in \mathcal{F}(\Pi V_j, W): \phi \text{ é $k$-linear}\}\)
\end{definition}

\begin{note}
    Se \(\forall j\leq k\), \(V_j =V\) então \(\hom^k(\Pi V_j,W) = \hom^k(V,W)\). Também, se  \(W = \F \), chamamos seus elementos de \emph{formas $k$-lineares} em \(V\). 
\end{note}

\alerta{
    \centering \(U \leq \Pi V_j\  \textcolor{red}{\nRightarrow} \ \phi(U) \leq W\)
    \tcblower
    \begin{example}
        Sejam \(V=\F^2\), \(W = \F^4\) e \(\phi : \F^2 \times \F^2 \ni ((x_1,y_1), (x_2,y_2))  \mapsto (x_1x_2, x_1y_2, y_1x_2, y_1y_2) \in \F^4\). \comentario{Veja que é bilinear, logo que \(\operatorname{Im}(\phi) = \{(a_j): a_1a_4 = a_2a_3\}\), pegue dois caras \(u,v \in \operatorname{Im}(\phi) : u+v \notin \operatorname{Im}(\phi)\) } 
    \end{example}
}

\theoremnum{9.1.3.}
\begin{theorem}
    \label{thm:9.1.3}
    Pra cada \(j\leq k\) sejam \(\alpha_j = (v_{i,j})_{i \in I_j}\) uma base de \(V_j\), \(I = \Pi^k I_j\) e \((w_\textbf{i})_{\textbf{i} \in I}\), familia de vetores em \(W\). Então, \(\exists ! \phi \in \hom^k(\Pi V_j, W)\) tal que \(\forall \textbf{i} \in I\) temos \(\phi((v_{\textbf{i}, j})) = w_\textbf{i}\). 
\end{theorem}

\comentario{A demostração é completamente análoga da feita pra \(\blacksquare\) \ref{thm:6.1.6}, mudando, claro, o não despreciavél negôcio da gestão dos índices}

\propositionnum{9.1.4.}
\begin{proposition}
    Sejam \(V_j\), \(\alpha_j\) como no \(\blacksquare\) \ref{thm:9.1.3}, \(\beta = (w_s)_{s\in S}\) uma base de \(W\), \(\textbf{v}_\textbf{i}:= (v_{\textbf{i},j})_{\textbf{i} \in I}\) e \(\forall (\textbf{i}, s) \in I \times S\) seja \(\phi_{\textbf{i}, s} \in \hom^k(\Pi V_j, W)\) o único tal que \(\forall \textbf{i}' \in I, \phi_{\textbf{i},s}(v_{\textbf{i}'}) = \delta_{\textbf{i}, \textbf{i}'} w_s\). Então \((\phi_{\textbf{i}, s})\) é l.i., mas ainda, se \(\forall j\leq k, \dim V_j < \infty \) então \((\phi_{\textbf{i}, s})\) é base do \(\hom^k(\Pi V_j, W)\). 
\end{proposition}

\demo{
    dsakfj
}